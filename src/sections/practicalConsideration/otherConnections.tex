\subsection{Anderweitige Zusammenhänge}\label{subsec:other-connections}

Neben den bisher bereits mehrfach angesprochenen
\glqq Mandelbrot Videos\grqq\footnote{
  Auch unter \glqq Zoomvideos\grqq~oder \glqq Bilderfahrten\grqq~zu finden
}, die ebenfalls einen digitalen Zusammenhang zu der Mandelbrot-Menge besitzen
und die trotz ihres Primärziels, das normalerweise die Unterhaltung ist, eine
ausgesprochene Komplexität besonders hinsichtlich ihrer Optimierungsmöglichkeiten,
aufweisen, denen man problemlos eine eigene Arbeit dedizieren könnte,
gibt es, wie in der Einleitung erwähnt,
viele biologische Zusammenhänge zwischen der Natur und Fraktalen:
Obwohl es über den Umfang dieser Arbeit hinausgehen würde, auf diesen Bereich
der Biologie genauer einzugehen, so soll dennoch angemerkt werden,
dass Fraktale wie die Mandelbrot-Menge der Natur ermöglichen,
komplexe Anordnungen und Konstruktionen möglichst ressourcen- und platzschonend
darzustellen, da Informationen sowohl für beispielsweise den Baumstamm als auch
einen kleinen Ast aufgrund der, wie bei Satelliten der Mandelbrot-Menge auftretenden,
Selbstähnlichkeit von Fraktalen nur einmal gespeichert werden müssen.