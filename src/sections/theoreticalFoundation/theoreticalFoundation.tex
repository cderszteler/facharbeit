\newsection{Theoretische Grundlage}{theoretical-foundation}

Die von Benoît Mandelbrot \hyperref[app:4]{[A.4]} entdeckte und nach ihm benannte
Mandelbrot-Menge befasst sich mit der bereits im vorherigen Kapitel vorgestellen,
komplexen Iteration \(z_{n+1} = z_n^2 + c \text{ mit } z_0 = 0\) und einem variablen
Wert für \(c\)~\cite*[S.25]{schuh_fraktale_2017}.
Die Mandelbrot-Menge enthält dabei alle Werte dieser Iteration, die beschränkt sind.
Mathematisch wird die Menge wie folgt definiert:

\[
  \mathbb{M} = \{c \in \mathbb{C} \; |\;  \forall n \in \mathbb{N}:\; |f_c^n(z)|\; \leqslant 2;\, n \to \infty\}
  \quad
  \text{mit}
  \quad
  f_c(z) = z^2 + c;\, z,c \in \mathbb{C}
\]