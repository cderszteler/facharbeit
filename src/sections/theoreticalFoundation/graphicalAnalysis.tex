\subsection{Grafische Analyse}\label{subsec:graphical-analysis}

Im Folgenden soll der grundlegende Aufbau der in einem kartesischen Diagramm
entstehenden Formation der Mandelbrot-Menge erörtert als auch eine
Erklärung zur Darstellungsweise und der im Späteren aufkommenden
Farbkodierung gegeben werden.

% TODO: Replace image with own
\bigskip
\begin{figure}[h!]\label{fig:mandelbrot-set}
  \centering
  \includegraphics[width=\textwidth]{images/exampleImage}
  \caption[Caption for LOF]{
    Exemplarische Darstellung der Mandelbrot-Menge\footnotemark
  }
\end{figure}
% TODO: Add hyperref to code in appendix
\footnotetext{Generiert durch den Code $\cdots$}

Die hier zu sehende Grafik entspricht der Darstellung der Mandelbrot-Menge
in einer komplexen Zahlenebene und wird aufgrund seiner Form
\glqq Apfelm\"annchen\grqq~genannt.
Neben der großen, einheitlichen Struktur in der Mitte, lassen sich,
zum Beispiel im negativen Teil der reellen Achse,
ebenfalls deutlich kleinere, ähnlich aussende Formationen um den
eigentlichen Hauptkörper, dem Apfelm\"anchen, erkennen.
Diese werden \textbf{Satelliten} genannt und existieren aufgrund der Selbstähnlichkeit
der Mandelbrot-Menge in unendlicher Stückzahl - und zwar nicht nur für den Hauptkörper,
sondern auch für jeden Satelliten selbst~\cite{lomonaco_quasi-conformal_2018}.

\subsubsection{Exemplarische Kartografierung}

Die entstehenden Formationen der Mandelbrot-Menge, die teilweise erst bei sehr
kleinen Ausschnitten erkennbar sind, sind kartografiert und teilweise
wegen einer gewissen Ähnlichkeit nach Objekten aus der realen Welt benannt.
So bezeichnet man die größte kreisförmige Kardioide oder auch \glqq Knospe\grqq
~als \glqq K\"orper\grqq~(wobei dieser genauer unterteilt werden kann)
und die daran angrenzende Kardioide als
\glqq Kopf\grqq\footnote{Vgl. \hyperref[app:5]{A.5}}.

Obwohl man jeden Punkt beziehungsweise jeden Ausschnitt einzeln beliebig detailliert
analysieren kann, werden aufgrund der Selbstähnlichkeit Elemente mit ähnlichem
oder gleichen Aufbau erneut aufkehren und dementsprechend gleich benannt.
Im Folgenden soll beispielhaft ein Ausschnitt des in diesem
Video~\cite{beyer_zoomfahrt_2017} gezeigten
\glqq Tal der Seepferdchen\grqq~analysiert werden:

Die Spalte zwischen Kopf und K\"orper wird \glqq Tal der Seepferdchen\grqq
~genannt\footnote{Vgl. \hyperref[app:6.1]{A.6.1}}
~\cite{robert_p_seahorse_2010}, da bei Vergrößerung dieses Ausschnitts
sich unter anschaulicher Farbkodierung auf der rechten Seite
Seepferdchen-ähnliche Formationen erkennen lassen\footnote{Vgl. \hyperref[app:6.2]{A.6.2}}.
Vergrößert man die Sicht auf das Seepferdchen-Tal stark, so lassen sich,
neben weiteren (teils deformierten) Satelliten\footnote{Vgl. \hyperref[app:6.3]{A.6.3}},
bei genauerer Betrachtung des \glqq Seepferdchenschwanzes\grqq~ein
Misiurewicz-Punkt erkennen\footnote{Vgl. \hyperref[app:6.4]{A.6.4}}.
Dieser Misiurewicz-Punkt zeigt ebenfalls die Selbstähnlichkeit der Mandelbrot-Menge auf,
da dieser Punkt sich neben einer Drehung kaum von der eigentlichen
Mandelbrot-Menge unterscheidet~\cite{lei_similarity_1989}.
Vergrößert man diesen Punkt weiter, so findet man erneut einen im Vergleich
zum Apfelm\"annchen sehr ähnlichen aussehenden
Satelliten\footnote{Vgl. \hyperref[app:6.5]{A.6.5}}.

\subsubsection{Farbbedeutung und -kodierung}