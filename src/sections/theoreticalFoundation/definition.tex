\subsection{Mathematische Definition}\label{subsec:mathematical-definition}

Die Mandelbrot-Menge befasst sich mit der bereits im vorherigen Kapitel vorgestellen,
komplexen Iteration \(z_{n+1} = z_n^2 + c \text{ mit } z_0 = 0\) und einem variablen
Wert für \(c\)~\cite*[S.25]{schuh_fraktale_2017}.
Die Mandelbrot-Menge enthält dabei alle komplexen Werte für $c$, mit denen die
oben angegebene Iteration beschränkt ist.
Mathematisch lässt sich die Menge iterativ wie folgt definiert:

\begin{equation}\label{eq:mathematical-definition}
  \mathbb{M} = \{c \in \mathbb{C} \; |\;  \forall n \in \mathbb{N}:\; |f_c^n(z)|\; \leqslant 2;\, n \to \infty\}
  \quad
  \text{mit}
  \quad
  f_c(z) = z^2 + c;\, z,c \in \mathbb{C}
\end{equation}

Wie in der Definition zu sehen, wird der Funktionswert der
gegen Unendlich strebenden $n$-ten Iteration absolut betrachtet, da die
Funktion symmetrisch zur reellen Achse ist.

Ebenfalls zu betrachten ist die Einschränkung auf Funktionswerte $\leqslant 2$.
Dies hängt mit einem Kreis, dessen Mittelpunkt im Ursprung liegt und dessen
Radius 2 beträgt, zusammen. % TODO: Add [Vgl. A.4] picture
Für alle Funktionswerte, die sich in einer Iteration ergeben und außerhalb dieses Radius
beziehungsweise dem Kreis liegen, lässt das jeweilige $c$ aus der Mandelbrot-Menge
ausschließen~\cite{munafo_escape_1997}.
Obwohl der gesamte Beweis dessen den Rahmen dieser Arbeit sprengen würde,
so soll dennoch angemerkt werden, dass mithilfe der Dreiecksungleichung und
vollständiger Induktion unter der Vorausnahme von $|z_n| > 2 \text{ und } |z_n| > |c|$
folgende Ungleichung, die eine divergente Entwicklung repräsentiert,
$\frac{|z_{n+11}|}{|z_n|} > 1$ bewiesen werden kann, wobei zusätzlich gezeigt
werden kann, dass für alle Werte von $|c| > 2$ nach spätestens 2 Iterationen gilt:
$z_{n=2} = |c^2 + c| \geqslant |c|^2 - |c| > 2$.