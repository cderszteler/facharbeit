\subsection{Iterationen}\label{subsec:iterations}

Iterationen beziehen sich in der Mathematik auf das Wiederholen einer bestimmten
Prozedur beziehungsweise in diesem Fall einer Berechnung.
Bei Funktionsiterationen iteriert (also wiederholt) man die Berechnung eines
Funktionswerts mit dem Funktionsargument des vorherigen Funktionswerts:
$z_1 = f(z_0),\, z_2 = f(z_1),\, z_3 = f(z_2),\, \cdots,\, z_n = f(z_{n-1})$.

Eine wichtige Eigenschaft von Iterationen ist die Entwicklung von
$z \text{ für } z \to \infty$.
Dabei wird unterschieden, ob die Iteration divergent ist,
das heißt gegen Unendlich verläuft (\glqq explodiert\grqq),
oder sich einem bestimmten Punkt annähert.
Letzteres bezeichnet man als einen beschränkten Verlauf.

Letzter Verlauf ist bei Iterationen schwer vorauszusagen, denn ähnliche wirkende
Funktionen können dennoch sehr unterschiedliche Entwicklungen aufweisen.
Die Funktionen $f_c(z) = z^2 + c \text{ mit } z_0 = 0$ stellen beispielhaft
die unterschiedlichen Verlaufsformen für verschiedene Parameter $c$ mithilfe von
$c_1 = 1 \text{, } c_2 = -1 \text{ und } c_3 = 0.5$ dar:

\begin{table}[h!]
  \centering
  \begin{tabular}{@{}cccc@{}}
    \toprule
    & \multicolumn{3}{c}{$f_c(z)$ für unterschiedliche Parameter $c$} \\
    \cmidrule(lr){2-4}
    Iteration & $ c_1 = 1$ & $ c_2 = -1$ & $ c_3 = 0.5$ \\
    \midrule
    1. & $1$ & $-1$ & $0,5$ \\
    2. & $2$ & $0$ & $0,75$ \\
    3. & $\boldsymbol{5}$ & $-1$ & $1,0625$ \\
    4. & $26$ & $0$ & $\approx 1,6289 $ \\
    5. & $667$ & $-1$ & $\approx \boldsymbol{3,1533} $ \\
    6. & $\approx \num{1,9e11}\ $ & $0$ & $\approx 10,4433 $ \\
    7. & $\approx \num{3,9e22}\ $ & $-1$ & $\approx 109,5625 $ \\
    \bottomrule
  \end{tabular}
  \caption{
    $f_c(z) \text{ verläuft mit } c_1 \text{ und } c_3$ divergent,
    higegen ist der Verlauf von $f_c(z) \text{ mit } c_2$ beschränkt.
    Die dick markierten Zahlen sind für eine spätere Erwähnung dieser Tabelle
    relevant [Eigene Darstellung].
  }
  \label{tab:iterations-example}
\end{table}