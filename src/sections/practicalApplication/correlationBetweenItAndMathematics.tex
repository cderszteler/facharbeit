\subsection{Korrelation zwischen Informationstechnik und Mathematik}
\label{subsec:correlation-between-it-and-mathematics}

Die im letzten Kapitel analysierte \hyperref[fig:mandelbrot-set]
{Abbildung \ref{fig:mandelbrot-set}} wurde,
wie in der Beschriftung beschrieben, mit dem Code aus \hyperref[app:10]{A.10}
generiert.
Im Vergleich zu den mathematischen Überlegungen ist die
informationstechnische Herangehensweise dabei sehr ähnlich:

Grundlegend wird jeder Pixel mit seiner $x$- und $y$-Koordinate des zu generierenden
Ausschnitts einer komplexen Zahl zugeordnet.
Dabei entspricht (in der regulären Darstellung) die $x$-Koordinate dem Realteil
$a$ und die $y$-Koordinate dem Imaginärteil $b$, sodass eine komplexe Zahl
$z = a + bi$ in einem Pixel $P \text{ durch } x = a \text{ und } y = b$ ausgedrückt
werden kann.

Das naheliegendste Problem, neben vielen ausschließlich informationstechnischen
Optimierungsaufgaben, ergibt sich bei der Berechnung, ob es sich beim jeweiligen Pixel
beziehungsweise dem jeweiligen $c$, um ein Element in
$\mathbb{M}$ handelt.
Denn im Gegensatz zu der mathematischen Betrachtung ist es in der konkreten Umsetzung
nicht möglich, die Iterationsanzahl $n$ bei $f^n(z)$ gegen Unendlich konvergieren zu lassen.
Deshalb setzt man einen Grenzwert $m$ an Iterationsdurchläufen, ab dem,
sofern das zu überprüfende $c$ noch nicht aus der Mandelbrot-Menge
ausgeschlossen wurde, dieses als Element von $\mathbb{M}$ angenommen wird.

Folglich bestimmt dieser Wert indirekt die Auflösung beziehungsweise Genauigkeit
des zu generierenden Bilds und muss deshalb bei kleineren Ausschnitten besonders
hoch sein, da dabei ein Unterschied zwischen komplexen Zahlen ausgemacht werden
muss, dessen Werte sich lediglich um geringe Nachkommastellen unterscheiden.
Die Auswirkungen dieser Iterationsgrenze sind durch die Bildergalerie
\hyperref[app:8]{A.8}, die generierte Bilder der Mandelbrot-Menge
mit unterschiedlich (niedrigen) Grenzen darstellt, anschaulich visualisiert.