\subsection{Farbkodierung}\label{subsec:color-coding}

Um die in der \hyperref[subsubsec:color-meaning]{Sektion \ref{subsubsec:color-meaning}}
erklärten Farben zu kodieren, eignet sich zunächst das HSV (Hue, Saturation, Value)
Farbmodel, womit sich mithilfe von prozentualen Werten eine
Farbveränderung hinsichtlich sowohl der Helligkeit
als auch der Farbsättigung erzielen lässt.
Dies wird anhand des Verhältnisses der benötigten Iterationen $n$ zu der Iterationsgrenze $m$
für jeden zu überprüfenden komplexen Wert $c \text{ für } \mathbb{M}$
berechnet \cite{robert_p_color_2022}.
Als Beispiel sei Hue mit $186\degree$ und Value mit $100\%$ gegeben, woraus
sich für jedes $c$ folgendes HSV ergibt:
$ HSV(186\degree, \frac{n}{m} \cdot 100\%, 100\%)$.

Diese Farbkodierung ist aufgrund ihrer Simplizität in ihrer beschriebenen
Form direkt informationstechnisch umsetzbar, jedoch eignet sich solch eine
Herangehensweise aus Performance- beziehungsweise Optimierungsgründen nicht
für jede zu überprüfende, komplexe Zahl $c$, wenn es sich bei der
Farbzusammenstellung um einen komplexeren Zusammenhang wie in den Elementen
der Bildergalerie \hyperref[app:6]{A.6} handelt.
Um dieses Problem zu lösen, erstellt man sogenannte Colormaps
(engl.: Farbpaletten), wie zum Beispiel \hyperref[app:9.1]{\ref{app:9.1}}
aus den Werten des oberen Beispiels.
Für ein jeweiliges $c$ wird mithilfe solcher Farbpaletten eine Farbe erneut anhand
dessen Verhältnisses von benötigten Iterationen $n$ zu der Iterantionsgrenze $m$
ermittelt; dabei existieren jedoch bereits alle vorkommenden Farben, was somit
die Berechnung des aufwendigeren Farbgenerierungsprozesses\footnote{
  Also die Auswahl und korrekte Zusammenstellung der Farben einer Farbpalette.
}
von dem Berechnungsprozess der eigentlichen Mandelbrot-Menge entkoppelt.
In \hyperref[app:9.2]{\ref{app:9.2}} ist die für alle für die Arbeit generierten
Abbildungen benutzte Colormap zu sehen, die im Vergleich zur vorherigen
dargestellten Palette mehr als zwei (insgesamt 4) Ankerpunkte\footnote{
  Hier: Eindeutige HSV-Farben, die in gewissen Abständen auf einer Palette
  platziert werden und zwischen denen ein Gradient ensteht.
} besitzt, wobei diese, im Gegensatz zu einer komplexen\footnote{
  Eine Farbpalette mit vielen Ankerpunkten.
} Colormap \hyperref[app:9.3]{[z.B. \ref{app:9.3}]},
mit der man visuell anschaulichere Bilder generieren kann,
weiterhin recht minimal ist.