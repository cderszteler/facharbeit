\subsection{Digitale Bildgenerierung}\label{subsec:digital-generation}

Im weiteren Verlauf werden informationstechnische Aspekte
hinsichtlich der Bildgenerierung der Mandelbrot-Menge untersucht.
Dafür wird zunächst die Funktionsweise als auch die dabei durch die Unterschiede
zur reinen Mathematik entstehenden Probleme erläutert.
Im Anschluss werden verschiedene Herangehensweisen und Algorithmen zur Farbkodierung
exemplarisch vorgestellt.

\subsubsection{Korrelation zwischen Informationstechnik und Mathematik}
\label{subsubsec:correlation-between-it-and-mathematics}

Die im letzten Kapitel analysierte \hyperref[fig:mandelbrot-set]
{Abbildung \ref{fig:mandelbrot-set}} wurde,
wie in der Beschriftung beschrieben, mit dem Code aus \hyperref[app:10]{A.10}
generiert.
Im Vergleich zu den mathematischen Überlegungen ist die
informationstechnische Herangehensweise dabei sehr ähnlich:

Grundlegend wird jeder Pixel mit seiner $x$- und $y$-Koordinate des zu generierenden
Ausschnitts einer komplexen Zahl zugeordnet.
Dabei entspricht (in der regulären Darstellung) die $x$-Koordinate dem Realteil
$a$ und die $y$-Koordinate dem Imaginärteil $b$, sodass eine komplexe Zahl
$z = a + bi$ in einem Pixel $P \text{ durch } x = a \text{ und } y = b$ ausgedrückt
werden kann.

Das naheliegendste Problem, neben vielen ausschließlich informationstechnischen
Optimierungsaufgaben, ergibt sich bei der Berechnung, ob es sich beim jeweiligen Pixel
beziehungsweise dem jeweiligen $c$, um ein Element in
$\mathbb{M}$ handelt.
Denn im Gegensatz zu der mathematischen Betrachtung ist es in der konkreten Umsetzung
nicht möglich, die Iterationsanzahl $n$ bei $f^n(z)$ gegen Unendlich konvergieren zu lassen.
Deshalb setzt man einen Grenzwert $m$ an Iterationsdurchläufen, ab dem,
sofern das zu überprüfende $c$ noch nicht aus der Mandelbrot-Menge
ausgeschlossen wurde, dieses als Element von $\mathbb{M}$ angenommen wird.

Folglich bestimmt dieser Wert indirekt die Auflösung beziehungsweise Genauigkeit
des zu generierenden Bilds und muss deshalb bei kleineren Ausschnitten besonders
hoch sein, da dabei ein Unterschied zwischen komplexen Zahlen ausgemacht werden
muss, dessen Werte sich lediglich um geringe Nachkommastellen unterscheiden.
Die Auswirkungen dieser Iterationsgrenze sind durch die Bildergalerie
\hyperref[app:8]{A.8}, die generierte Bilder der Mandelbrot-Menge
mit unterschiedlich (niedrigen) Grenzen darstellt, anschaulich visualisiert.

\subsubsection{Farbkodierung}\label{subsubsec:color-coding}

Um die in der \hyperref[subsubsec:color-meaning]{Sektion \ref{subsubsec:color-meaning}}
erklärten Farben zu kodieren, eignet sich zunächst das HSV (Hue, Saturation, Value)
Farbmodel, womit sich mithilfe von prozentualen Werten eine
Farbveränderung hinsichtlich sowohl der Helligkeit
als auch der Farbsättigung erzielen lässt.
Dies wird anhand des Verhältnisses der benötigten Iterationen $n$ zu der Iterationsgrenze $m$
für jeden zu überprüfenden komplexen Wert $c \text{ für } \mathbb{M}$
berechnet \cite{robert_p_color_2022}.
Als Beispiel sei Hue mit $186\degree$ und Value mit $100\%$ gegeben, woraus
sich für jedes $c$ folgendes HSV ergibt:
$ HSV(186\degree, \frac{n}{m} \cdot 100\%, 100\%)$.

Diese Farbkodierung ist aufgrund ihrer Simplizität in ihrer beschriebenen
Form direkt informationstechnisch umsetzbar, jedoch eignet sich solch eine
Herangehensweise aus Performance- beziehungsweise Optimierungsgründen nicht
für jede zu überprüfende, komplexe Zahl $c$, wenn es sich bei der
Farbzusammenstellung um einen komplexeren Zusammenhang wie in den Elementen
der Bildergalerie \hyperref[app:6]{A.6} handelt.
Um dieses Problem zu lösen, erstellt man sogenannte Colormaps
(engl.: Farbpaletten), wie zum Beispiel \hyperref[app:9.1]{\ref{app:9.1}}
aus den Werten des oberen Beispiels.
Für ein jeweiliges $c$ wird mithilfe solcher Farbpaletten eine Farbe erneut anhand
dessen Verhältnisses von benötigten Iterationen $n$ zu der Iterantionsgrenze $m$
ermittelt; dabei existieren jedoch bereits alle vorkommenden Farben, was somit
die Berechnung des aufwendigeren Farbgenerierungsprozesses\footnote{
  Also die Auswahl und korrekte Zusammenstellung der Farben einer Farbpalette.
}
von dem Berechnungsprozess der eigentlichen Mandelbrot-Menge entkoppelt.
In \hyperref[app:9.2]{\ref{app:9.2}} ist die für alle für die Arbeit generierten
Abbildungen benutzte Colormap zu sehen, die im Vergleich zur vorherigen
dargestellten Palette mehr als zwei (insgesamt 4) Ankerpunkte\footnote{
  Hier: Eindeutige HSV-Farben, die in gewissen Abständen auf einer Palette
  platziert werden und zwischen denen ein Gradient ensteht.
} besitzt, wobei diese, im Gegensatz zu einer komplexen\footnote{
  Eine Farbpalette mit vielen Ankerpunkten.
} Colormap \hyperref[app:9.3]{[z.B. \ref{app:9.3}]},
mit der man visuell anschaulichere Bilder generieren kann,
weiterhin recht minimal ist.