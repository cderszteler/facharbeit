\subsection{Iterationen}\label{subsec:iterations}

Iterationen beziehen sich in der Mathematik auf das Wiederholen einer bestimmten
Prozedur beziehungsweise in diesem Fall einer Berechnung.
Bei Funktionsiterationen iteriert (also wiederholt) man die Berechnung eines
Funktionswerts mit dem Funktionsargument des vorherigen Funktionswerts:
$z_1 = f(z_0),\, z_2 = f(z_1),\, z_3 = f(z_2),\, \cdots,\, z_n = f(z_{n-1})$.

Eine wichtige Eigenschaft von Iterationen ist die Entwicklung von
$z \text{ für } z \to \infty$.
Dabei wird unterschieden, ob die Iteration divergent ist,
das heißt gegen Unendlich verl\"auft (\glqq explodiert\grqq),
oder sich einem bestimmten Punkt, ann\"ahert.
Letzteres bezeichnet man als einen beschränkten Verlauf.

Dieser Verlauf ist bei Iterationen, die ihre Ausgangswerte als neue Eingangswerte
benutzen, schwer vorauszusagen.
Dabei k\"onnen \"ahnlich Funktionen bereits sehr unterschiedliche Entwicklungen aufweisen.
Die Funktionen $f_c(z) = z^2 + c \text{ mit } z_0 = 0$ stellt beispielhaft
die unterschiedlichen Verlaufsformen für verschiedene Parameter $c$ durch
$c_1 = 1 \text{, } c_2 = -1 \text{ und } c_3 = 0.5$ dar:

\begin{table}[h!]
  \centering
  \begin{tabular}{@{}cccc@{}}
    \toprule
    & \multicolumn{3}{c}{$f_c(z)$ für unterschiedliche Parameter $c$} \\
    \cmidrule(lr){2-4}
    Iteration & $ c_1 = 1$ & $ c_2 = -1$ & $ c_3 = 0.5$ \\
    \midrule
    1. & $1$ & $-1$ & $0,5$ \\
    2. & $2$ & $0$ & $0,75$ \\
    3. & $\boldsymbol{5}$ & $-1$ & $1,0625$ \\
    4. & $26$ & $0$ & $\approx 1,6289 $ \\
    5. & $667$ & $-1$ & $\approx \boldsymbol{3,1533} $ \\
    6. & $\approx \num{1,9e11}\ $ & $0$ & $\approx 10,4433 $ \\
    7. & $\approx \num{3,9e22}\ $ & $-1$ & $\approx 109,5625 $ \\
    \bottomrule
  \end{tabular}
  \caption{
    $f_c(z) \text{ verl\"auft mit } c_1 \text{ und } c_3$ divergent,
    higegen ist der Verlauf f\"ur $f_c(z) \text{ mit } c_2$ beschr\"ankt.
    Die dick markierten Zahlen sind für eine spätere Erwähnung dieser Tabelle
    relevant.
  }
  \label{tab:iterations-example}
\end{table}