\subsection{Komplexe Zahlen}\label{subsec:complex-numbers}

Unter den komplexen Zahlen $\mathbb{C}$ versteht man die nächst größere Zahlenmenge
nach den reellen Zahlen $\mathbb{R}$,
die zus\"atzlich zu einem Realteil auch einen sogenannten Imagin\"arteil besitzen.
Sie werden im weiteren Verlauf in der kartesischen Form
\(z = a + bi\) dargestellt, wobei \(a\) der Realteil und \(bi\) der Imagin\"arteil ist.
Der Buchstabe \(i\) steht hierbei für die imaginäre Einheit und
ist definiert durch die Gleichung \(i^2 = -1\).

\subsubsection{Multiplikation \& Addition von komplexen Zahlen}
\label{subsubsec:addition-and-multiplication-of-complex-numbers}

Viele Rechenoperationen mit komplexen Zahlen funktionieren anders, als man sie
von den reellen oder natürlichen Zahlen gewohnt ist.

Zur Addition zwei komplexer Zahlen addiert man den
Realteil und den Imagin\"arteil getrennt voneinander und fügt diesen
danach wieder zusammen~\cite{lichtenegger_komplexe_2002}:
\((a_1 + {b_1}i) + (a_2 + {b_2}i) = a_1 + a_2 + (b_1 + b_2)i\).

Um komplexe Zahlen zu multiplizieren, wendet man das Distributivgesetz an,
indem man den zweiten Faktor ebenfalls in seinen Realteil und seinen
Imagin\"arteil trennt und diese jeweils einzeln mit dem ersten Faktor multipliziert
~\cite{lichtenegger_komplexe_2002}.
Die zwei entstehenden Produkte lassen sich dann wie oben beschrieben addieren.
Bei der Multiplikation mit dem Imagin\"arteil multipliziert man unter anderem
zwei imagin\"are Elemente miteinander.
Da \(i^2 = -1\) gilt, entsteht durch diese Multiplikation
ein negatives, aber reales Produkt.
Wie in \hyperref[app:1]{A.1} gezeigt, gilt somit:
\( (a + bi)(c + di) = ac - bd + (bc + ad)i \)

Das für die Mandelbrot-Menge besonders wichtige Quadrieren von komplexen Zahlen
lässt sich mit der kartesischen Form ebenfalls herleiten \hyperref[app:2]{[A.2]}.
Für eine gegeben, zu quadrierende, komplexe Zahl \(a + bi\) gilt somit:
\(a^2 - b^2 + 2abi\)


Ein illustriertes Beispiel soll beide Rechenoperationen veranschaulichen:

\begin{equation}\label{eq:complex-numbers-example}
  \begin{split}
    (\begingroup\color{red} -3\endgroup + \begingroup\color{blue} 6i\endgroup)^2
      + (\begingroup\color{red} 7\endgroup + \begingroup\color{blue} (-4i)\endgroup) \\
    = (
        (\begingroup\color{red} -3\endgroup \cdot \begingroup\color{red} (-3)\endgroup)
          - (\begingroup\color{red} 6\endgroup \cdot \begingroup\color{red} 6\endgroup)
        + (
          (\begingroup\color{blue} 6\endgroup \cdot \begingroup\color{blue} (-3)\endgroup)
          + (\begingroup\color{blue} -3\endgroup \cdot \begingroup\color{blue} 6\endgroup)
        )i
      )
      + (\begingroup\color{red} 7\endgroup + \begingroup\color{blue} (-4i)\endgroup) \\
    = (\begingroup\color{red} -27\endgroup + \begingroup\color{blue} (-36i)\endgroup)
      + (\begingroup\color{red} 7\endgroup + \begingroup\color{blue} (-4i)\endgroup) \\
    = \begingroup\color{red} -20\endgroup + \begingroup\color{blue} (-40i)\endgroup
  \end{split}
\end{equation}

\subsubsection{Graphische Darstellung komplexer Zahlen}

Komplexe Zahlen können wie auch Zahlen anderer Zahlenmengen grafisch
dargestellt werden.
Da komplexe Zahlen jedoch sowohl aus einem Realteil und einem Imagin\"arteil
bestehen, reicht eine Gerade nicht aus, um diese darzustellen, stattdessen
braucht man eine \textbf{Ebene}\footnote{
  Ebenfalls unter komplexer Zahlenebene und gaußscher Zahlenebene zu finden.
}.
Diese komplexe Zahlenebene teilt den Realteil auf der waagerechten Achse und
den Imagin\"arteil auf die horizontale Achse auf.
Eine komplexe Zahl \(z_1 = a + bi\) besitzt somit die Koordinaten \( P(a|b)\).


\begin{figure}[H]\label{fig:complex-numbers-figure-example}
  \begin{center}
    \scalebox{.5}{
      \begin{tikzpicture}
        \begin{scope}[thick,font=\scriptsize]
          \draw [->] (-4,0) -- (4,0) node [above left]  {$Re$};
          \draw [->] (0,-4) -- (0,4) node [below right] {$Im$};

          \foreach \n in {-3,...,-1,1,2,...,3}{
            \draw (\n, 3pt) -- (\n, -3pt)   node [below] {\(\n\)};
            \draw (3pt,\n) -- (-3pt,\n)   node [left] {\(\n i\)};
          }

          \draw [color=black, fill=black] (0,0) circle(0.05);
          \draw [color=black, fill=black] (2,3) circle(0.05);
          \draw [color=black, fill=black] (-3, -2) circle(0.05);
          \node [color=black] at (0.75,0.25) {\( P_1(0|0i)\)};
          \node [color=black] at (2.75,3.25) {\( P_2(2|3i)\)};
          \node [color=black] at (-2.25,-1.75) {\( P_3(-3|2i)\)};
        \end{scope}
      \end{tikzpicture}
    }
    \caption{
      Komplexe Ebene mit den Punkten \( P_1, P_2 \text{ und } P_3\)
      \hyperref[app:3]{[A.3]}
    }
  \end{center}
\end{figure}