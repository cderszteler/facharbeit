\subsection{Digitale Bild- und Videogenerierung}\label{subsec:digital-generation}

\subsubsection{Korrelation zwischen Informationstechnik und Mathematik}
\label{subsubsec:correlation-between-it-and-mathematics}

Die im letzten Kapitel vorgestellte und analysierte \hyperref[fig:mandelbrot-set]
{Abbildung \ref{fig:mandelbrot-set}} wurde,
wie es in der Beschriftung steht, mit dem Code aus \hyperref[app:last]{[A.Last]}
generiert.
Im Vergleich zu den theoretischen Überlegungen, ist die
informationstechnische Herangehensweise dabei sehr ähnlich:

Grundlegend wird jeder Pixel mit seiner $x$- und $y$-Koordinate des zu generierenden
Ausschnitts einer komplexen Zahl zugeordnet.
Dabei entspricht (in der regulären Darstellung) die $x$-Koordinate dem Realteil
$a$ und die $y$-Koordinate dem Imagin\"arteil $b$, sodass eine komplexe Zahl
$c = a + bi$ in einem Pixel $P \text{ durch } x = a \text{ und } y = b$ ausgedrückt
werden kann.

Das naheliegendste und offensichtlichste Problem,
neben vielen rein informationstechnischen Optimierungsaufgaben,
ergibt sich bei der Berechnung, ob es sich beim jeweiligen Pixel
beziehungsweise bei der jeweiligen komplexen Zahl, um ein Element aus
$\mathbb{M}$ handelt.
Denn im Gegensatz zu der mathematischen Betrachtung, die die Iterationsanzahl
$n$ bei $f^n(z)$ gegen unendlich laufen lässt, ist dies in der konkreten Umsetzung
nicht möglich.
Deshalb setzt man einen Grenzwert an Iterationsdurchläufen, ab dem, sofern der zu
überprüfende Wert für $c$ noch nicht aus der Mandelbrot-Menge ausgeschlossen wurde,
dieser als Element von $\mathbb{M}$ angenommen wird.

Folglich bestimmt dieser Wert indirekt die Auflösung beziehungsweise Genauigkeit
des zu generierenden Bilds und muss deshalb bei kleineren Ausschnitten besonders
hoch sein, da dabei ein Unterschied zwischen komplexen Zahlen ausgemacht werden
muss, dessen Werte sich lediglich um geringe Nachkommastellen unterscheiden.
Die Auswirkungen dieser Iterationsgrenze sind durch die Bildergalerie
\hyperref[app:8]{A.8}, die generierte Bilder der Mandelbrot-Menge
mit unterschiedlich (niedrigen) Grenzen darstellt, anschaulich visualisiert.