\newsection{Einleitung}{introduction}

Die Mandelbrot-Menge ist durch ihre hübschen, ansehnlichen Darstellungen,
verglichen mit anderen mathematischen Phänomenen, recht bekannt.
Dies liegt nicht allein an ihrer visuellen Attraktivität,
sondern vielmehr auch an Benoît Mandelbrot \hyperref[app:1]{[A.1]},
dem Entdecker dieser Menge.
Dieser sorgte mit seinen häufigen Vorträgen und Büchern dafür, dass sich Fraktale,
also selbstähnliche\footnote{
  Das heißt, sich selbst wiederholend oder in ähnlicher Form erneut aufkommend.
}, geometrische Figuren mit gebrochener Dimension\footnote{
  Im Vergleich zu zum Beispiel einem zwei-dimensionalen Viereck.
},
vornehmlich die Mandelbrot-Menge, in der Bevölkerung weit verbreiteten
\cite[Vgl. letzten Absatz]{ibm_fractal_2011}.

Obwohl die Natur mit ihren fraktal-ähnlichen Formationen wie dem Aufbau einer
Schneeflocke, dem Verlauf eines Flusses oder die Verteilung von Baumästen
~\cite{nnart_fractals_nodate} die Inspiration für Mandelbrot war~\cite{zink_kosmische_2014},
so liegt der Ursprung dieser Arbeit in den für
manchen simpler erscheinenden, viel moderneren aber dennoch genauso spannenden,
computer-generierten Videos\footcite[Vgl. bspw.][]{maths_town_eye_2017},
die man im Internet finden kann.
Mit unter anderem der Frage, wie diese Videos in Ansätzen generiert werden können,
und vielem weiteren beschäftigt sich diese Arbeit.

Dafür und zum vollen Verständnis der Mandelbrot-Menge ist Grundlagenwissen
gewisser Themengebiete erforderlich, das in Kapitel 2 näher erörtert wird.
Kapitel 3 beschäftigt sich daraufhin mit der mathematischen Betrachtung der
Mandelbrot-Menge und insbesondere mit der Analyse visueller Darstellungen dieser.
Abschließend befasst sich diese Arbeit in Kapitel 4 mit der praktischen Anwendung
der Mandelbrot-Menge in Form von Bildgenerierungen mithilfe von Computern.