\section{Einleitung}\label{sec:introduction}

Die Mandelbrot-Menge ist verglichen mit anderen mathematischen Phänomenen
äußerst bekannt, was nicht nur an ihrer visuellen Attraktivität liegt,
sondern vielmehr auch an Benoît Mandelbrot beziehungsweise seinen Intentionen selbst.
Dieser sorgte mit seinen vielen Vorträgen und Büchern dafür, dass sich Fraktale,
vornehmlich die Mandelbrot-Menge, in der Bevölkerung weit verbreiteten
~\footcite[Vgl. letzten Absatz]{ibm_fractal_2011}.
Die Schönheit der Mandelbrot-Menge ist unbestreitbar, jedoch steckt mehr in dieser
als nur ansehnliche, sich wiederholende Muster.

Obwohl die Natur mit ihren fraktal-ähnlichen Formationen wie dem Aufbau einer
Schneeflocke, dem Verlauf eines Flusses oder die Verteilung von Baumästen
~\cite{nnart_fractals_nodate} die Inspiration für Mandelbrot war
~\cite{zink_kosmische_2014}, so liegt der Ursprung dieser Arbeit in den für
manche simpler erscheinenden, viel moderneren aber dennoch genauso spannenden,
computer-generierten Videos\footcite[Vgl. bspw.][]{maths_town_eye_2017},
die man im Internet finden kann.
Mit unter anderem der Frage, wie diese Videos in Ansätzen generiert werden können
und vielem weiteren beschäftigt sich diese Arbeit.

Dafür jedoch und zum vollen Verständnis der Mandelbrot-Menge ist Wissen außerhalb
des regulären Schullehrplans erforderlich, das in Kapitel 2 näher erörtert werden.
Kapitel 3 beschäftigt sich daraufhin mit der Mandelbrot-Menge selbst und insbesondere
mit der Analyse visueller Darstellungen dieser.
Abschließend befasst sich diese Arbeit in Kapitel 4 mit der praktischen Anwendung
der Mandelbrot-Menge in Form von Bildgenerierungen mithilfe von Computern als auch
anderweitigen Zusammenhänge zwischen dem theoretischen, mathematischen Konzept
der Mandelbrot-Menge und der realen Welt.