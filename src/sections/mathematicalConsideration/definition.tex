\subsection{Definition}\label{subsec:definition}

Die Mandelbrot-Menge $\mathbb{M}$ wird mit der bereits im vorherigen Kapitel vorgestellen,
komplexen Iteration $z_{n+1} = z_n^2 + c \text{ mit } z_0 = 0$ und einem variablen
Wert für $c$~\cite*[S.477ff.]{weitz_konkrete_2018} definiert.
Dabei enthält die Menge alle komplexen Werte für $c$, mit denen die
oben angegebene Iteration beschränkt ist.
Mathematisch ist die Menge iterativ wie folgt definiert:

\begin{equation}\label{eq:mathematical-definition}
  \mathbb{M} = \{c \in \mathbb{C} \; |\;  \forall n \in \mathbb{N}:\; |f_c^n(z)|\; \leqslant 2\}
  \quad
  \text{mit}
  \quad
  f_c(z) = z^2 + c;\, z \in \mathbb{C}
\end{equation}

Wie in der Definition zu sehen, wird der Funktionswert der
gegen Unendlich strebenden $n$-ten Iteration, ausgedrückt durch $f^n(z)$,
absolut betrachtet, was bedeutet, dass die Funktion symmetrisch zur reellen Achse ist.

Ebenfalls zu betrachten ist die Einschränkung auf Funktionswerte $\leqslant 2$, denn
für alle Funktionswerte, die sich in der oben genannten Iteration ergeben
und $> 2$ sind, lässt sich das jeweilige $c$ aus der Mandelbrot-Menge
ausschließen.
Obwohl der gesamte Beweis dessen über den Rahmen dieser Arbeit hinausginge,
so soll dennoch angemerkt werden, dass mithilfe der Dreiecksungleichung und
vollständiger Induktion unter der Vorausnahme von $|z_n| > 2 \text{ und } |z_n| > |c|$
folgende Ungleichung, die eine divergente Entwicklung repräsentiert,
$\frac{|z_{n+1}|}{|z_n|} > 1$ bewiesen werden kann~\cite{munafo_escape_1997},
wobei sich zusätzlich zeigen lässt, dass für alle Werte von $|c| > 2$
nach spätestens 2 Iterationen gilt: $z_2 = |c^2 + c| \geqslant |c|^2 - |c| > 2$.

Es befinden sich deshalb alle Werte für $c$ als auch somit die grafische Darstellung
der Mandelbrot-Menge in einem Kreis, dessen Mittelpunkt im Ursprung liegt,
mit dem Radius 2 [Vgl. \hyperref[app:4]{A.4}].