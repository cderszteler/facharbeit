\subsection{Iterationen}\label{subsec:iterations}

Iterationen beziehen sich in der Mathematik auf das Wiederholen einer bestimmten
Prozedur beziehungsweise in diesem Fall einer Berechnung.
Bei Funktionsiterationen iteriert (also wiederholt) man die Berechnung eines
Funktionswerts mit dem Funktionsargument des vorherigen Funktionswerts:
\(z_1 = f(z_0),\, z_2 = f(z_1),\, z_3 = f(z_2),\, \cdots,\, z_n = f(z_{n-1})\)

Eine wichtige Eigenschaft von
Iterationen ist das Verhalten für \(z \to \infty\) .
Dabei wird unterschieden, ob die Iteration gegen Unendlich verläuft
(\glqq explodiert\grqq ) oder sich einem bestimmten Punkt, oftmals Null, ann\"ahert.
Letzteres bezeichnet man als einen beschränkten Verlauf.

Dieser Verlauf ist bei Iterationen, die ihre Ausgangswerte als neue Eingangswerte
benutzen, schwer vorauszusagen.
Dabei k\"onnen \"ahnlich Funktionen bereits sehr unterschiedliche Verhalten aufweisen.
Die Funktionen \(f(z) = z^2 + c \text{ mit }\ z_0 = 0\) stellt beispielhaft
die unterschiedlichen Verlaufstypen für \(c_1 = 1 \text{ und }\ c_2 = -1\) dar:

\begin{table}[h!]\label{tab:iterations-example}
  \centering
  \begin{tabular}{@{}ccc@{}}
    \toprule
    & \multicolumn{2}{c}{Parameter für \(f(z)\)} \\
    \cmidrule(lr){2-3}
    Iteration & \( c_1 = 1\) & \( c_2 = -1\) \\
    \midrule
    1. & 1 & -1 \\
    2. & 2 & 0 \\
    3. & 5 & -1 \\
    4. & 26 & 0 \\
    5. & 667 & -1 \\
    6. & \(\approx \num{1,9e11}\ \) & 0 \\
    7. & \(\approx \num{3,9e22}\ \) & -1 \\
    \bottomrule
  \end{tabular}
\end{table}