\subsection{Komplexe Zahlen}\label{subsec:complex-numbers}

Unter komplexen Zahlen versteht man die nächst größere Zahlenmenge nach den reellen
Zahlen $\mathbb{R}$, die zus\"atzlich zu einem Realteil auch einen Imagin\"arteil
besitzen.
Sie werden im weiteren Verlauf in der kartesischen Form
\(z = a + bi\) dargestellt, wobei \(a\) der Realteil und \(bi\) der Imagin\"arteil ist.
Der Buchstabe \(i\) steht hierbei für die imaginäre Einheit und
ist definiert durch die Gleichung \(i^2 = -1\).

\subsubsection{Addition von komplexen Zahlen}\label{subsubsec:addition-of-complex-numbers}

Viele Rechenoperationen mit komplexen Zahlen funktionieren anders, als man sie
von den reellen oder natürlichen Zahlen gewohnt ist.
Die Addition ist jedoch im Vergleich zu anderen Operationen wie der Division
noch recht simpel.
Zur Addition zwei komplexer Zahlen addiert man den
\begingroup\color{red}Realteil\endgroup~und den
\begingroup\color{blue}Imagin\"arteil\endgroup
~getrennt voneinander und fügt diesen danach wieder zusammen:
\((a_1 + {b_1}i) + (a_2 + {b_2}i) = a_1 + a_2 + (b_1 + b_2)i\)
~\cite{lichtenegger_komplexe_2002}.
Ein illustriertes Beispiel, gegeben sei \(z_1 = -3 + 6i\) und \(z_2 = 2 + 0.5i\):

\begin{equation}\label{eq:complex-numbers-addition-example}
  z_1 + z_2
  = (\begingroup\color{red} -3\endgroup + \begingroup\color{blue} 6i\endgroup)
    + (\begingroup\color{red} 2\endgroup + \begingroup\color{blue} 0.5i\endgroup)
  = \begingroup\color{red} (-3 + 2)\endgroup + \begingroup\color{blue} (6 + 0.5)\endgroup i
  = -1 + 6.5i
\end{equation}