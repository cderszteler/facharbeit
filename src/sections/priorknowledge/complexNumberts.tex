\subsection{Komplexe Zahlen}\label{subsec:complex-numbers}

Unter komplexen Zahlen versteht man die nächst größere Zahlenmenge nach den reellen
Zahlen $\mathbb{R}$, die zus\"atzlich zu einem Realteil auch einen Imagin\"arteil
besitzen.
Sie werden im weiteren Verlauf in der kartesischen Form
\(z = a + bi\) dargestellt, wobei \(a\) der Realteil und \(bi\) der Imagin\"arteil ist.
Der Buchstabe \(i\) steht hierbei für die imaginäre Einheit und
ist definiert durch die Gleichung \(i^2 = -1\).

\subsubsection{Addition von komplexen Zahlen}\label{subsubsec:addition-of-complex-numbers}

Viele Rechenoperationen mit komplexen Zahlen funktionieren anders, als man sie
von den reellen oder natürlichen Zahlen gewohnt ist.
Zur Addition zwei komplexer Zahlen addiert man den
\begingroup\color{red}Realteil\endgroup~und den
\begingroup\color{blue}Imagin\"arteil\endgroup
~getrennt voneinander und fügt diesen danach wieder zusammen
~\cite{lichtenegger_komplexe_2002}:
\((a_1 + {b_1}i) + (a_2 + {b_2}i) = a_1 + a_2 + (b_1 + b_2)i\).
Ein illustriertes Beispiel, gegeben sei \(z_1 = -3 + 6i\) und \(z_2 = 2 + 0.5i\):

\begin{equation}\label{eq:complex-numbers-addition-example}
  z_1 + z_2
  = (\begingroup\color{red} -3\endgroup + \begingroup\color{blue} 6i\endgroup)
    + (\begingroup\color{red} 2\endgroup + \begingroup\color{blue} 0.5i\endgroup)
  = \begingroup\color{red} (-3 + 2)\endgroup + \begingroup\color{blue} (6 + 0.5)\endgroup i
  = -1 + 6.5i
\end{equation}

\subsubsection{Multiplikation von komplexen Zahlen}\label{subsubsec:multiplication-of-complex-numbers}

Die Multiplikation funktioniert ebenfalls nicht so, wie man sie von den
reellen Zahlen gewohnt ist.
Um komplexe Zahlen zu multiplizieren, bedient man sich stattdessen dem Distributivgesetz,
indem man den zweiten Faktor erneut in seinen
\begingroup\color{red}Realteil\endgroup~und seinen
\begingroup\color{blue}Imagin\"arteil\endgroup
~trennt und diese jeweils einzeln mit dem ersten Faktor multipliziert
~\cite{lichtenegger_komplexe_2002}.
Die zwei entstehenden Produkte lassen sich dann wie oben beschrieben addieren.
Bei der Multiplikation mit dem Imagin\"arteil multipliziert man unter anderem
zwei imagin\"are Elemente miteinander.
Da \(i^2 = -1\) gilt, entsteht durch diese Multiplikation
ein negatives, aber reales Produkt.
Dadurch gilt:

\begin{equation}\label{eq:complex-numbers-multiplication}
  \begin{split}
    (a + bi)(c + di) \\
    = \begingroup\color{red} c\endgroup (a + bi)
      + \begingroup\color{blue} di\endgroup (a + bi) \\
    = \begingroup\color{red} ac\endgroup
      + \begingroup\color{blue} bci\endgroup
      + \begingroup\color{blue} adi\endgroup
      + \begingroup\color{red} bdi^2\endgroup \\
    = \begingroup\color{red} ac\endgroup
      + \begingroup\color{blue} bci\endgroup
      + \begingroup\color{blue} adi\endgroup
      \textbf{ - } \begingroup\color{red} bdi\endgroup \\
    = \begingroup\color{red} ac - bd\endgroup
      + \begingroup\color{blue}(bc + ad)i\endgroup \\
  \end{split}
\end{equation}

Ein weiteres, illustriertes Beispiel, gegeben sei erneut
\(z_1 = -3 + 6i\) und \(z_2 = 2 + 0.5i\):

\begin{equation}\label{eq:complex-numbers-multiplication-example}
  \begin{split}
    z_1 \cdot z_2 \\
    = (\begingroup\color{red} -3\endgroup + \begingroup\color{blue} 6i\endgroup)
      \cdot (\begingroup\color{red} 2\endgroup + \begingroup\color{blue} 0.5i\endgroup) \\
    = \begingroup\color{red} (-3 \cdot 2 - 6 \cdot 0.5)\endgroup
      + \begingroup\color{blue} (6 \cdot 2 + (-3) \cdot 0.5)\endgroup i \\
    = \begingroup\color{red} (-6 - 3)\endgroup
      + \begingroup\color{blue} (12 - 1.5)\endgroup i \\
    = \begingroup\color{red} -9\endgroup + \begingroup\color{blue} 10.5i\endgroup
  \end{split}
\end{equation}

Das für die Mandelbrot-Menge besonders wichtige Quadrieren von komplexen Zahlen
lässt sich mit der kartesischen Form ebenfalls herleiten.
Für eine gegeben, zu quadrierende, komplexe Zahl \(z_1\) gilt:

\begin{multline}\label{eq:complex-numbes-squaring}
  z_1^2
    = z_1 + z_1
    = (a + bi) \cdot (a + bi)
    = a \cdot (a + bi) + bi \cdot (a + bi) \\
    = a^2 + abi + abi - b^2
    = a^2 - b^2 + 2abi
\end{multline}