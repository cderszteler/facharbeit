\newsection{Fazit}{conclusion}

Zusammenfassend lässt sich festhalten, dass durch die verhältnismäßig simple
Gleichung $z_{n+1} = z_n^2 + c$ sich eine riesige Welt der Fraktale eröffnet
hat, die durch ihre visuellen Darstellungen mit ihrer atemberaubenden Schönheit
nicht nur eine Nische an Mathematikern, sondern seit langem erneut auch die
breitere Gesellschaft erreichen und erstaunen konnte.
Diese Arbeit hat einen grundlegenden Überblick über die Mathematik hinter der
Mandelbrot-Menge geliefert als auch eine Erklärung und die damit einhergehenden
Komplikationen zur Generierung solcher faszinierenden Bildern dargelegt.

Nichtsdestotrotz ist der abgedeckte Bereich dieser Abhandlung begrenzt, denn es
existieren unzählige weitere Zusammenhänge zwischen der Mandelbrot-Menge und
anderen mathematischen Phänomenen wie Pi oder der Fibonacci-Folge.
Es würde sich ebenfalls nicht um eine vollständige Arbeit über die
Mandelbrot-Menge handeln, wenn nicht die Julia-Mengen und ihre enge Verknüpfung
zu der als übergeordnet zu betrachtenden Mandelbrot-Menge erwähnt werden würde.

Abschließend lässt sich mit Sicherheit sagen, dass man gespannt abwarten kann,
welche interessanten Entdeckungen diesem jungen Gebiet der Mathematik folgen
werden.